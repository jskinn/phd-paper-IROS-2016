% !TEX root = root.tex

\subsection{Simulation}

The use of simulation in robotics is not a new idea. Popular robot simulators such as Gazebo \cite{Koenig2004} and Player/Stage \cite{Gerkey2003} have existed for many years, and are commonly used to test robot motion and control. The use of game engines for robot simulation is also not new, with the USARsim project based on the Unreal Tournament 3 engine \cite{Carpin2007}, and other projects using the Unity engine \cite{mattingly2012robot}. Nobody has yet made use of modern, high-fidelity engines such as Unity 5 or Unreal Engine 4 (the successor to the Unreal Tournament 3 engine).

The more specific field of robotic vision has not seen much application of simulation at all, it instead prefers to work from image datasets captured from the real world.  The key requirement for simulation in computer vision is that the simulator is capable of photo-realistic rendering and lighting, which has historically been out of reach for older platforms, including the Unreal Tournament 3 engine or Gazebo. Unreal Engine 4 however has powerful tools for realistic materials and lighting \cite{karis2013real}, which make it possible to create simulated environment that produce images similar to the real world.

Unreal Engine 4 has a number of additional advantages that make it suitable as a simulator platform for computer vision. It is developed and maintained by Epic Games Inc, and uses physics and modelling tools from nVidia, which allow for complex and interactive dynamic environments. It also allows full access to it's source code, allowing a stimulation designer complete control over the simulation. It is also free for non-commercial uses, including research.

\subsection{Place Recognition}

\begin{itemize}
    \item Sequence SLAM
    \begin{itemize}
        \item Developed by Milford and Wyeth... (citation)
        \item Performs place recognition using 
    \end{itemize}
    \item {Challenges}
        \begin{itemize}
            \item challenges include viewpoint invariant place recognition, condition invariant place recognition, place recognition in dynamic environments
            \item 
        \end{itemize}
\end{itemize}